\documentclass[12pt]{article}

\usepackage[left=2cm,top=2cm,right=2cm, bottom=2cm]{geometry}
\usepackage[T1]{fontenc}
\usepackage[utf8]{inputenc}
\usepackage[spanish]{babel}
\usepackage{amsfonts,setspace}
\usepackage{amsmath}
\usepackage{amssymb, amsmath, amsthm}
\usepackage{comment}
\usepackage{amssymb}
\usepackage{dsfont}
\usepackage{anysize}
\usepackage{multicol}
\usepackage{enumerate}
\usepackage{graphicx}
\usepackage{fancyhdr}
\usepackage{wasysym}
\usepackage{enumerate}
\usepackage{enumitem}
\usepackage{hyperref}

\pagestyle{fancy}

\theoremstyle{plain}
\newtheorem{teo}{Teorema}
\newtheorem{lemma}{Lemma}
\newtheorem{prop}{Proposici\'on}
\newtheorem{cor}{Corolario}
\theoremstyle{definition}
\newtheorem{defi}{Definici\'on}
\newtheorem{eje}{Ejemplo}
\newtheorem{obs}{Observaci\'on}
\newtheorem{propi}{Propiedades}

\newcommand{\norm}[1]{\lVert#1\rVert}
\newcommand{\ds}{\displaystyle}
\newcommand{\R}{\mathbb{R}}

% Error fixes
\makeatletter
\newcommand\RedeclareMathOperator{%
  \@ifstar{\def\rmo@s{m}\rmo@redeclare}{\def\rmo@s{o}\rmo@redeclare}%
}
% this is taken from \renew@command
\newcommand\rmo@redeclare[2]{%
  \begingroup \escapechar\m@ne\xdef\@gtempa{{\string#1}}\endgroup
  \expandafter\@ifundefined\@gtempa
     {\@latex@error{\noexpand#1undefined}\@ehc}%
     \relax
  \expandafter\rmo@declmathop\rmo@s{#1}{#2}}
% this is just \@declmathop without \@ifdefinable
\newcommand\rmo@declmathop[3]{%
  \DeclareRobustCommand{#2}{\qopname\newmcodes@#1{#3}}%
}
\@onlypreamble\RedeclareMathOperator
\makeatother

\DeclareMathOperator{\sen}{sen}
\RedeclareMathOperator{\cos}{cos}
\RedeclareMathOperator{\tan}{tan}
\RedeclareMathOperator{\sec}{sec}
\DeclareMathOperator{\cosec}{cosec}
\DeclareMathOperator{\cotan}{cotan}
\DeclareMathOperator{\arcsen}{arcsen}
\RedeclareMathOperator{\arccos}{arccos}
\RedeclareMathOperator{\arctan}{arctan}

\DeclareMathOperator{\senh}{senh}
\RedeclareMathOperator{\cosh}{cosh}
\RedeclareMathOperator{\tanh}{tanh}
\DeclareMathOperator{\sech}{sech}
\DeclareMathOperator{\cosech}{cosech}
\DeclareMathOperator{\cotanh}{cotanh}
\DeclareMathOperator{\arcsenh}{arcsenh}
\DeclareMathOperator{\arccosh}{arccosh}
\DeclareMathOperator{\arctanh}{arctanh}

\DeclareMathOperator{\Dom}{Dom}
\DeclareMathOperator{\Rec}{Rec}
\RedeclareMathOperator{\Im}{Im}

\DeclareMathOperator{\Int}{Int}
\DeclareMathOperator{\Adh}{Adh}
\DeclareMathOperator{\Fr}{Fr}
\DeclareMathOperator{\co}{co}

\DeclareMathOperator{\dist}{dist}
\DeclareMathOperator*{\argmin}{arg\,min}
\DeclareMathOperator*{\argmax}{arg\,max}

\let\lim=\undefined\DeclareMathOperator*{\lim}{\text{lím}}
\let\max=\undefined\DeclareMathOperator*{\max}{\text{máx}}
\let\min=\undefined\DeclareMathOperator*{\min}{\text{mín}}
\let\inf=\undefined\DeclareMathOperator*{\inf}{\text{ínf}}

\newcommand{\pint}[2]{\left< #1,#2\right>}
\newcommand{\ssi}{\Longleftrightarrow}
\newcommand{\imp}{\Longrightarrow}
\newcommand{\pmi}{\Longleftarrow}
\newcommand{\ipartial}[2]{\dfrac{\partial #1}{\partial #2}}
\newcommand{\ider}[2]{\dfrac{d #1}{d #2}}
\newcommand{\iipartial}[2]{\dfrac{\partial^2 #1}{\partial #2^2}}
\newcommand{\iider}[2]{\dfrac{d^2 #1}{d #2^2}}
\newcommand{\ijpartial}[3]{\dfrac{\partial^2 #1}{\partial #2 \partial #3}}

\newcommand{\N}{\mathbb{N}}
\newcommand{\Z}{\mathbb{Z}}
\newcommand{\Q}{\mathbb{Q}}
\newcommand{\C}{\mathbb{C}}
\newcommand{\K}{\mathbb{K}}
\renewcommand{\P}{\mathbb{P}}

\newcommand{\A}{\mathcal{A}}
\newcommand{\B}{\mathcal{B}}
\newcommand{\D}{\mathcal{D}}
\newcommand{\E}{\mathcal{E}}
\newcommand{\F}{\mathcal{F}}
\renewcommand{\H}{\mathcal{H}}
\newcommand{\I}{\mathcal{I}}
\renewcommand{\L}{\mathcal{L}}
\newcommand{\M}{\mathcal{M}}
\newcommand{\Partes}[1]{\mathcal{P}(#1)}
\renewcommand{\S}{\mathcal{S}}
\newcommand{\Tau}{\mathcal{T}}
\newcommand{\V}{\mathcal{V}}
\newcommand{\X}{\mathcal{X}}
\newcommand{\vacio}{\emptyset}
\newcommand{\tsc}{\textsc}

\newcommand{\header}{
    \fancyhead[L]{Facultad de Ciencias Físicas y Matemáticas}
	\fancyhead[R]{Universidad de Chile}
	\vspace*{-1cm}
	\hspace{-0.5cm}\begin{minipage}{0.7\textwidth}
	\begin{flushleft}
	\textbf{MA3705 Algoritmos Combinatoriales}.\\
	\textbf{Profesor:} Iván Rapaport.\\
	\textbf{Auxiliares:} Antonia Labarca y Cristian Palma.\\
	\end{flushleft}
	\end{minipage}
	\begin{minipage}{0.3\textwidth}
	\begin{flushright}
	\hspace{-0.5cm}\includegraphics[scale=0.2]{img/dim.pdf}
	\end{flushright}
	\end{minipage}

	\medskip
}
\begin{document}
\header
\begin{center}
	\LARGE \bf{Auxiliar 4}
\end{center}

\begin{center}
	\bf{Dicaminos de Costo Mínimo}\\
\end{center}

\begin{enumerate}[label ={\bf P\arabic*}]
	\item Considere un grafo dirigido $G = (V,E)$ y dos vértices $s,t \in V$. Usted desea viajar desde el nodo $s$ hasta
	el nodo $t$, lamentablemente atravesar una ariste $e$ no es necesariamente seguro, por lo que tiene una
	probabilidad $p_e \in [0, 1]$ de sobrevivir el viaje por la arista $e$.
	Considere dichas probabilidades son independientes, por lo que la probabilidad de sobrevivir a un $st$-camino $P$ está
	dada por $\prod_{e\in P} p_e$. Deseamos encontrar el camino que maximice la probabilidad de sobrevivir.
	Reduzca el problema a uno de camino de costo mínimo y dé un algoritmo para resolverlo.

	\item Con el uso masivo de GPS, las calles que forman parte de los caminos más cortos se están volviendo cada vez más
	congestionadas, limitando la utilidad de seguir las indicaciones dadas por el dispositivo. Por ello, le interesa calcular
	el camino casi más corto entre su casa y su trabajo. El camino casi más corto entre $s$ y $t$ se define como el $st$-camino
	más corto que no usa ninguna arista de algún camino de costo mínimo entre $s$ y $t$.
	
	\begin{enumerate}
		\item Asuma que todo $st$-camino mínimo es disjunto. Diseñe un algoritmo que encuentre
		el camino casi más corto entre $s$ y $t$ en $O(|E|(|E|+|V|\log|V|))$.
		\item Para el caso general, diseñe un algoritmo en $O(|E|+|V|\log|V|)$ considerando verificar para cada arista $uv$ si esta pertenece a algún camino más corto comprobando si la suma de la distancia de $s$ a $u$, del largo de $uv$ y de la distancia de $v$ a $t$ es igual a la distancia máxima.
	\end{enumerate}

	\item \textbf{[Propuesto] Maximum Bottleneck Path:} Sea $G = (V,E)$ un digrafo, $c: E \to \R$ una función de capacidades y dados $s, t \in V$,
	modifique el algoritmo Dijkstra para encontrar el $st$-camino cuya arista con menos capacidad sea de capacidad máxima.
	En otras palabras, se define la capacidad de un camino como la capacidad mínima de sus aristas y se busca el $st$-camino de capacidad máxima.

	\item \textbf{[Propuesto]} Sea $G = (V,E)$ un grafo y suponga que tiene acceso a una matriz $D$ tal que $D_{u,v}$ es el costo del camino de costo
	mínimo entre $u$ y $v$. Suponga que ahora se disminuye el costo $w_e$ de una arista $e \in E$ a un nuevo costo $w'_e$.
	Diseñe un algoritmo $O(|V|^2)$ para actualizar la matriz $D$ considerando el nuevo costo.

	\item \textbf{[Propuesto]} Sea $G = (V,E)$ un grafo con largos $\ell: E \to \R_+$ y tiempos de espera $w: V \to \R_+$ en los nodos.
	El largo de un camino $P$ de $s$ a $t$ se define como la suma de los largos de los arcos más la suma de los tiempos de
	espera de los nodos que atravieza. Encuentre un algoritmo que calcule una arborescencia de caminos mínimos
	desde $s$ vía una reducción.

\end{enumerate}
\end{document}