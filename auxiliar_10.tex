\documentclass[12pt]{article}

\usepackage[left=2cm,top=2cm,right=2cm, bottom=2cm]{geometry}
\usepackage[T1]{fontenc}
\usepackage[utf8]{inputenc}
\usepackage[spanish]{babel}
\usepackage{amsfonts,setspace}
\usepackage{amsmath}
\usepackage{amssymb, amsmath, amsthm}
\usepackage{comment}
\usepackage{amssymb}
\usepackage{dsfont}
\usepackage{anysize}
\usepackage{multicol}
\usepackage{enumerate}
\usepackage{graphicx}
\usepackage{fancyhdr}
\usepackage{wasysym}
\usepackage{enumerate}
\usepackage{enumitem}
\usepackage{hyperref}

\pagestyle{fancy}

\theoremstyle{plain}
\newtheorem{teo}{Teorema}
\newtheorem{lemma}{Lemma}
\newtheorem{prop}{Proposici\'on}
\newtheorem{cor}{Corolario}
\theoremstyle{definition}
\newtheorem{defi}{Definici\'on}
\newtheorem{eje}{Ejemplo}
\newtheorem{obs}{Observaci\'on}
\newtheorem{propi}{Propiedades}

\newcommand{\norm}[1]{\lVert#1\rVert}
\newcommand{\ds}{\displaystyle}
\newcommand{\R}{\mathbb{R}}

% Error fixes
\makeatletter
\newcommand\RedeclareMathOperator{%
  \@ifstar{\def\rmo@s{m}\rmo@redeclare}{\def\rmo@s{o}\rmo@redeclare}%
}
% this is taken from \renew@command
\newcommand\rmo@redeclare[2]{%
  \begingroup \escapechar\m@ne\xdef\@gtempa{{\string#1}}\endgroup
  \expandafter\@ifundefined\@gtempa
     {\@latex@error{\noexpand#1undefined}\@ehc}%
     \relax
  \expandafter\rmo@declmathop\rmo@s{#1}{#2}}
% this is just \@declmathop without \@ifdefinable
\newcommand\rmo@declmathop[3]{%
  \DeclareRobustCommand{#2}{\qopname\newmcodes@#1{#3}}%
}
\@onlypreamble\RedeclareMathOperator
\makeatother

\DeclareMathOperator{\sen}{sen}
\RedeclareMathOperator{\cos}{cos}
\RedeclareMathOperator{\tan}{tan}
\RedeclareMathOperator{\sec}{sec}
\DeclareMathOperator{\cosec}{cosec}
\DeclareMathOperator{\cotan}{cotan}
\DeclareMathOperator{\arcsen}{arcsen}
\RedeclareMathOperator{\arccos}{arccos}
\RedeclareMathOperator{\arctan}{arctan}

\DeclareMathOperator{\senh}{senh}
\RedeclareMathOperator{\cosh}{cosh}
\RedeclareMathOperator{\tanh}{tanh}
\DeclareMathOperator{\sech}{sech}
\DeclareMathOperator{\cosech}{cosech}
\DeclareMathOperator{\cotanh}{cotanh}
\DeclareMathOperator{\arcsenh}{arcsenh}
\DeclareMathOperator{\arccosh}{arccosh}
\DeclareMathOperator{\arctanh}{arctanh}

\DeclareMathOperator{\Dom}{Dom}
\DeclareMathOperator{\Rec}{Rec}
\RedeclareMathOperator{\Im}{Im}

\DeclareMathOperator{\Int}{Int}
\DeclareMathOperator{\Adh}{Adh}
\DeclareMathOperator{\Fr}{Fr}
\DeclareMathOperator{\co}{co}

\DeclareMathOperator{\dist}{dist}
\DeclareMathOperator*{\argmin}{arg\,min}
\DeclareMathOperator*{\argmax}{arg\,max}

\let\lim=\undefined\DeclareMathOperator*{\lim}{\text{lím}}
\let\max=\undefined\DeclareMathOperator*{\max}{\text{máx}}
\let\min=\undefined\DeclareMathOperator*{\min}{\text{mín}}
\let\inf=\undefined\DeclareMathOperator*{\inf}{\text{ínf}}

\newcommand{\pint}[2]{\left< #1,#2\right>}
\newcommand{\ssi}{\Longleftrightarrow}
\newcommand{\imp}{\Longrightarrow}
\newcommand{\pmi}{\Longleftarrow}
\newcommand{\ipartial}[2]{\dfrac{\partial #1}{\partial #2}}
\newcommand{\ider}[2]{\dfrac{d #1}{d #2}}
\newcommand{\iipartial}[2]{\dfrac{\partial^2 #1}{\partial #2^2}}
\newcommand{\iider}[2]{\dfrac{d^2 #1}{d #2^2}}
\newcommand{\ijpartial}[3]{\dfrac{\partial^2 #1}{\partial #2 \partial #3}}

\newcommand{\N}{\mathbb{N}}
\newcommand{\Z}{\mathbb{Z}}
\newcommand{\Q}{\mathbb{Q}}
\newcommand{\C}{\mathbb{C}}
\newcommand{\K}{\mathbb{K}}
\renewcommand{\P}{\mathbb{P}}

\newcommand{\A}{\mathcal{A}}
\newcommand{\B}{\mathcal{B}}
\newcommand{\D}{\mathcal{D}}
\newcommand{\E}{\mathcal{E}}
\newcommand{\F}{\mathcal{F}}
\renewcommand{\H}{\mathcal{H}}
\newcommand{\I}{\mathcal{I}}
\renewcommand{\L}{\mathcal{L}}
\newcommand{\M}{\mathcal{M}}
\newcommand{\Partes}[1]{\mathcal{P}(#1)}
\renewcommand{\S}{\mathcal{S}}
\newcommand{\Tau}{\mathcal{T}}
\newcommand{\V}{\mathcal{V}}
\newcommand{\X}{\mathcal{X}}
\newcommand{\vacio}{\emptyset}
\newcommand{\tsc}{\textsc}

\newcommand{\header}{
    \fancyhead[L]{Facultad de Ciencias Físicas y Matemáticas}
	\fancyhead[R]{Universidad de Chile}
	\vspace*{-1cm}
	\hspace{-0.5cm}\begin{minipage}{0.7\textwidth}
	\begin{flushleft}
	\textbf{MA3705 Algoritmos Combinatoriales}.\\
	\textbf{Profesor:} Iván Rapaport.\\
	\textbf{Auxiliares:} Antonia Labarca y Cristian Palma.\\
	\end{flushleft}
	\end{minipage}
	\begin{minipage}{0.3\textwidth}
	\begin{flushright}
	\hspace{-0.5cm}\includegraphics[scale=0.2]{img/dim.pdf}
	\end{flushright}
	\end{minipage}

	\medskip
}
\begin{document}
\header
\begin{center}
	\LARGE \bf{Auxiliar 10}
\end{center}

\begin{center}
	\bf{Programas Lineales Enteros}\\
\end{center}

\begin{enumerate}[label ={\bf P\arabic*}]
	\item \textbf{Machine Productivity}

	En una fábrica, se tienen $n$ tanques de combustible distintos de tamaño $p_j, j\in[n]$ y $m$ máquinas cuya producción es proporcional al combustible administrado hasta cierta capacidad $k_i, i\in [m]$. Es posible cargar una máquina con combustible por sobre su capacidad, pero el exceso no generará ningún beneficio.
	
	Formule el problema de encontrar una asignación de tanques a máquinas que máximice la productividad de las máquinas como un programa lineal entero.
	
	\item \textbf{Symmetric Traveling Salesman Problem}
	
	Sea $G=(V, E)$ grafo no dirigido y $d: E \rightarrow \R_+$ una función donde $d_e$ representa la distancia entre los extremos de la arista $e$.
	Sea $x \in \{0,1\}^{E}$ la indicatriz de un tour (ciclo Hamiltoniano) en $G$.
	
	% Claramente, para todo $v \in V$, todo tour de $G$ contiene dos aristas incidentes en $v$.
	% Luego, $x$ debe satisfacer las siguientes restricciones:

	% \begin{equation}\tag{Restricciones de grado}
	% x\left(\delta_{E}(v)\right)=2, \quad \forall v \in V
	% \end{equation}

	% Además, como un tour es conexo y visita todos los vértices de $G$, cualquiera sea el corte $\emptyset \subsetneq S \subsetneq V$ en $G$ deben haber al menos dos aristas del tour cuyos extremos estén uno en $S$ y otro en su complemento. Es decir, $x$ debe satisfacer las siguientes restricciones:
	% \begin{equation}\tag{Restricciones de subtour}
	% x\left(\delta_{E}(S)\right) \geq 2, \quad \forall \emptyset \subsetneq S \subsetneq V
	% \end{equation}

	Pruebe que la formulación de Dantzig-Fulkerson-Johnson es exacta, es decir que un tour óptimo en $G$ está determinado por una solución óptima del siguiente PLE:

	\[\begin{array}{llr}
	(TSP)\quad 	&\displaystyle\min \sum_{e \in E} d_e x_e &\\
	s.a. \quad  & x(\delta_{E}(v))=2 & \forall v \in V\\
				& x(\delta_{E}(S)) \geq 2 & \forall \emptyset \subsetneq S \subsetneq V\\
				& x_e \in \{0,1\} & \forall e \in E
	\end{array}\]

	\item El principio del palomar establece que el problema no tiene solución:

	\textit{(P) Colocar $n+1$ palomas en $n$ agujeros de manera que no haya dos palomas que compartan un agujero.}
	
	Formule \textit{(P)} como un programa lineal entero con dos tipos de restricciones
	\begin{enumerate}
		\item Cada paloma debe entrar en un agujero.
		\item Para cada pareja de palomas, a lo sumo una de las dos aves puede entrar en un agujero dado.
	\end{enumerate}
	
	Demostrar que no existe ninguna solución entera que satisfaga $(a)$ y $(b)$, pero que el programa lineal con las restricciones $(a)$ y $(b)$ es factible.

	\item \textbf{[Propuesto] Knapsack}

	Supongamos que se tiene una mochila que puede llevar un máximo de peso $b$ y $n$ objetos donde el $i$-ésimo tiene peso $\omega_i$ y beneficio $c_i$.
	
	Nuestro objetivo será cargar la mochila con estos objetos sin violar la restricción de capacidad de la mochila de manera que maximicemos nuestra utilidad.

	\begin{enumerate}
		\item Modele el problema como un programa lineal entero y defina la relajación lineal.
	
		\item Demuestre que si $\bar{x}$ es un óptimo para el problema relajado, entonces a lo más una de sus coordenadas es fraccional (no es entera).
	\end{enumerate}

	\item \textbf{[Propuesto] Minimum Spanning Tree}

	Queremos construir una red de comunicación que conecte a todas las ciudades a costo mínimo. Para ello, contamos con un grafo no dirigido $G=(V,E)$, en donde $V$ es el conjunto de ciudades, $E$ las autopistas que conectan las ciudad y $\omega_e$ el costo de usar la autopista $e\in E$. El problema anterior se puede formular como el de encontrar un MST.
	Considere las siguientes formulaciones como programa lineal entero.
    
    \begin{align*}
        (M_1) \quad & \min \sum_{e\in E}\omega_ex_e\\
        s.a.\quad & x(E)=|V|-1\\
        & x(\delta(S))\geq 1,\forall S\subset V, S\neq \emptyset\\
        &x_e\in\{0,1\}\quad\forall e\in E
    \end{align*}
    
    \begin{align*}
        (M_2) \quad & \min \sum_{e\in E}\omega_ex_e\\
        s.a.\quad & x(E)=|V|-1\\
        & x(E(S))\leq |S|-1,\forall S\subset V, S\neq \emptyset\\
        &x_e\in\{0,1\}\quad\forall e\in E
    \end{align*}

	\begin{enumerate}
		\item Demuestre que ambos modelos con exactos.
		\item Relaje la integralidad de la primera formulación y dé un ejemplo de un punto factible que no corresponde a solución del problema original.
	\end{enumerate}
\end{enumerate}
\end{document}