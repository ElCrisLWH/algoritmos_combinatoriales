\documentclass[12pt]{article}

\usepackage[left=2cm,top=2cm,right=2cm, bottom=2cm]{geometry}
\usepackage[T1]{fontenc}
\usepackage[utf8]{inputenc}
\usepackage[spanish]{babel}
\usepackage{amsfonts,setspace}
\usepackage{amsmath}
\usepackage{amssymb, amsmath, amsthm}
\usepackage{comment}
\usepackage{amssymb}
\usepackage{dsfont}
\usepackage{anysize}
\usepackage{multicol}
\usepackage{enumerate}
\usepackage{graphicx}
\usepackage{fancyhdr}
\usepackage{wasysym}
\usepackage{enumerate}
\usepackage{enumitem}
\usepackage{hyperref}

\pagestyle{fancy}

\theoremstyle{plain}
\newtheorem{teo}{Teorema}
\newtheorem{lemma}{Lemma}
\newtheorem{prop}{Proposici\'on}
\newtheorem{cor}{Corolario}
\theoremstyle{definition}
\newtheorem{defi}{Definici\'on}
\newtheorem{eje}{Ejemplo}
\newtheorem{obs}{Observaci\'on}
\newtheorem{propi}{Propiedades}

\newcommand{\norm}[1]{\lVert#1\rVert}
\newcommand{\ds}{\displaystyle}
\newcommand{\R}{\mathbb{R}}

% Error fixes
\makeatletter
\newcommand\RedeclareMathOperator{%
  \@ifstar{\def\rmo@s{m}\rmo@redeclare}{\def\rmo@s{o}\rmo@redeclare}%
}
% this is taken from \renew@command
\newcommand\rmo@redeclare[2]{%
  \begingroup \escapechar\m@ne\xdef\@gtempa{{\string#1}}\endgroup
  \expandafter\@ifundefined\@gtempa
     {\@latex@error{\noexpand#1undefined}\@ehc}%
     \relax
  \expandafter\rmo@declmathop\rmo@s{#1}{#2}}
% this is just \@declmathop without \@ifdefinable
\newcommand\rmo@declmathop[3]{%
  \DeclareRobustCommand{#2}{\qopname\newmcodes@#1{#3}}%
}
\@onlypreamble\RedeclareMathOperator
\makeatother

\DeclareMathOperator{\sen}{sen}
\RedeclareMathOperator{\cos}{cos}
\RedeclareMathOperator{\tan}{tan}
\RedeclareMathOperator{\sec}{sec}
\DeclareMathOperator{\cosec}{cosec}
\DeclareMathOperator{\cotan}{cotan}
\DeclareMathOperator{\arcsen}{arcsen}
\RedeclareMathOperator{\arccos}{arccos}
\RedeclareMathOperator{\arctan}{arctan}

\DeclareMathOperator{\senh}{senh}
\RedeclareMathOperator{\cosh}{cosh}
\RedeclareMathOperator{\tanh}{tanh}
\DeclareMathOperator{\sech}{sech}
\DeclareMathOperator{\cosech}{cosech}
\DeclareMathOperator{\cotanh}{cotanh}
\DeclareMathOperator{\arcsenh}{arcsenh}
\DeclareMathOperator{\arccosh}{arccosh}
\DeclareMathOperator{\arctanh}{arctanh}

\DeclareMathOperator{\Dom}{Dom}
\DeclareMathOperator{\Rec}{Rec}
\RedeclareMathOperator{\Im}{Im}

\DeclareMathOperator{\Int}{Int}
\DeclareMathOperator{\Adh}{Adh}
\DeclareMathOperator{\Fr}{Fr}
\DeclareMathOperator{\co}{co}

\DeclareMathOperator{\dist}{dist}
\DeclareMathOperator*{\argmin}{arg\,min}
\DeclareMathOperator*{\argmax}{arg\,max}

\let\lim=\undefined\DeclareMathOperator*{\lim}{\text{lím}}
\let\max=\undefined\DeclareMathOperator*{\max}{\text{máx}}
\let\min=\undefined\DeclareMathOperator*{\min}{\text{mín}}
\let\inf=\undefined\DeclareMathOperator*{\inf}{\text{ínf}}

\newcommand{\pint}[2]{\left< #1,#2\right>}
\newcommand{\ssi}{\Longleftrightarrow}
\newcommand{\imp}{\Longrightarrow}
\newcommand{\pmi}{\Longleftarrow}
\newcommand{\ipartial}[2]{\dfrac{\partial #1}{\partial #2}}
\newcommand{\ider}[2]{\dfrac{d #1}{d #2}}
\newcommand{\iipartial}[2]{\dfrac{\partial^2 #1}{\partial #2^2}}
\newcommand{\iider}[2]{\dfrac{d^2 #1}{d #2^2}}
\newcommand{\ijpartial}[3]{\dfrac{\partial^2 #1}{\partial #2 \partial #3}}

\newcommand{\N}{\mathbb{N}}
\newcommand{\Z}{\mathbb{Z}}
\newcommand{\Q}{\mathbb{Q}}
\newcommand{\C}{\mathbb{C}}
\newcommand{\K}{\mathbb{K}}
\renewcommand{\P}{\mathbb{P}}

\newcommand{\A}{\mathcal{A}}
\newcommand{\B}{\mathcal{B}}
\newcommand{\D}{\mathcal{D}}
\newcommand{\E}{\mathcal{E}}
\newcommand{\F}{\mathcal{F}}
\renewcommand{\H}{\mathcal{H}}
\newcommand{\I}{\mathcal{I}}
\renewcommand{\L}{\mathcal{L}}
\newcommand{\M}{\mathcal{M}}
\newcommand{\Partes}[1]{\mathcal{P}(#1)}
\renewcommand{\S}{\mathcal{S}}
\newcommand{\Tau}{\mathcal{T}}
\newcommand{\V}{\mathcal{V}}
\newcommand{\X}{\mathcal{X}}
\newcommand{\vacio}{\emptyset}
\newcommand{\tsc}{\textsc}

\newcommand{\header}{
    \fancyhead[L]{Facultad de Ciencias Físicas y Matemáticas}
	\fancyhead[R]{Universidad de Chile}
	\vspace*{-1cm}
	\hspace{-0.5cm}\begin{minipage}{0.7\textwidth}
	\begin{flushleft}
	\textbf{MA3705 Algoritmos Combinatoriales}.\\
	\textbf{Profesor:} Iván Rapaport.\\
	\textbf{Auxiliares:} Antonia Labarca y Cristian Palma.\\
	\end{flushleft}
	\end{minipage}
	\begin{minipage}{0.3\textwidth}
	\begin{flushright}
	\hspace{-0.5cm}\includegraphics[scale=0.2]{img/dim.pdf}
	\end{flushright}
	\end{minipage}

	\medskip
}
\begin{document}
\header
\begin{center}
	\LARGE \bf{Auxiliar 2}
\end{center}

\begin{center}
	\bf{Árboles Generadores de Costo Mínimo}\\
\end{center}

\small

\begin{enumerate}[label ={\bf P\arabic*}]
	\item Considere la siguiente implementación del algoritmo de Prim-Jarnik:
	
	\begin{algorithm}
		\footnotesize
		\label{alg:1}
		\caption{\textit{Prim-Jarnik}.}
		\KwIn{$\langle G,w,r\rangle$ donde $G=(V,E)$ es un grafo conexo, $w:E \to \Q_+$ función de pesos y $r\in V$ vértice.}
		\KwOut{Árbol Generadores de Costo Mínimo de $G$}
		\For{$u \in V$}{
			$key(u) \gets \infty$\\
			$\pi(u) \gets \mathrm{NULL}$
		}
		$key(r) \gets 0$\\
		$Q \gets V$\\
		\While{$Q \neq \emptyset$}{
			$u \gets \textit{Extract-Min}(Q)$\\
			\For{$v\in N(u)$}{
				\If{$v\in Q$ y $w(uv)<key(u)$}{
					$\pi(v) \gets u$\\
					$key(v) \gets w(uv)$
				}
			}
		}
		$F\gets \{(v,\pi(v)) \mid v \in V-r\}$\\
		\Return{$(V,F)$}
	\end{algorithm}
	
	\begin{enumerate}
		\item Demuestre que el algoritmo es correcto.
		\item Calcule la complejidad del algoritmo.
	\end{enumerate}

	\item \textbf{Aproximación para TSP Métrico}
	
	Dado grafo completo $G$ y distancias $\ell: E(G) \rightarrow \R_{+}$ (satisface la desigualdad triangular), encontrar un ciclo Hamiltoniano de largo mínimo.

	Para resolver este problema usaremos \textbf{\textit{shortcut}}, estrategia que consiste en reemplazar dos aristas vecinas $v_1 v_2 v_3$ por una directa entre los vértices de los extremos $v_1 v_3$, saltandose así el vértice común $v_2$.

	Sea $\OPT$ el ciclo Hamiltoniano de largo mínimo.

	Sabiendo que todo multigrafo Euleriano tiene paseos Eulerianos, considere el siguiente algoritmo:
	
	\begin{algorithm}
		\footnotesize
		\label{alg:2}
		\caption{Aproximación para TSP Métrico.}
		\KwIn{$\langle G,\ell\rangle$ donde $G=(V,E)$ grafo completo y distancias $\ell: E(G) \rightarrow \R_{+}$.}
		\KwOut{Ciclo Hamiltoniano}
		$H \gets$ Submultigrafo Euleriano (conexo con grados pares) de $G$.\\
		$\ALG \gets$ Encontrar paseo Euleriano (recorre todas las aristas) para $H$.\\
		\While{$\exists u \in V, \delta_\ALG(u)>2$}{
			$\ALG \gets$ Aplicar \textit{shortcut} en $u$ a $\ALG$.\\
		}
		\Return{$\ALG$.}
	\end{algorithm}

	\begin{enumerate}
		\item Suponiendo que se puede generar el submultigrafo Euleriano $H$, demuestre que el algoritmo es correcto.
		Pruebe además que $\ell(\ALG) \leq \ell(H)$.
		\item Sea $T$ un MST de $G$ y sea $H$ el multigrafo que duplica las aristas de $T$. Pruebe que $\ell(H) \leq 2 \ell (\OPT)$.
		\item Sea $M$ un matching perfecto (aristas independientes que cubren todos los vértices) de peso mínimo de los vértices de grado impar de $T$.
		Un corolario del Lema del Apretón de Manos asegura que hay una cantidad par de dichos vértices, por lo que dicho matching perfecto existe.
		Pruebe que $H' = T + M$ es tal que $\ell(H') \leq \dfrac{3}{2}\ell(\OPT)$.
	\end{enumerate}

	\item \textbf{[Propuesto]} Considere la siguiente implementación del algoritmo de Boruvka:
	
	\begin{algorithm}
		\footnotesize
		\label{alg:3}
		\caption{\textit{Boruvka}.}
		\KwIn{$\langle G,c\rangle$ donde $G=(V,E)$ es un grafo conexo y $c:E \to \Q_+$ función de pesos.}
		\KwOut{Árbol Generadores de Costo Mínimo de $G$}
		$F\gets \emptyset$\\
		\While{$cc(V,F)\neq 1$}{
			Encontrar CC de $(V,F)$ y calcular $cc(V,F)$\\
			\For{U CC de $(V,F)$}{
				Encontrar $e_U \in \delta(U)$ arista de menor peso\\
				$F \gets F \cup {e_U}$
			}
		}
		\Return{$(V,F)$}
	\end{algorithm}
	
	\begin{enumerate}
		\item Justifique que solo basta probar que el grafo $(V, F)$ entregado es acíclico.
		
		Llame $G_{i}=\left(V, F_{i}\right)$ al grafo al comienzo de la iteración $i$ y $\mathcal{U}_{i}$ al conjunto de sus componentes conexas. Note que el algoritmo escoge para cada componente $U \in \mathcal{U}_{i}$ una arista $e_{U}$ de menor peso en $\delta(U)$ y llame $f\left(e_{U}\right) \in \mathcal{U}_{i}$ a la componente a la cual pertenece el extremo de $e_{U}$ que no está en $U$. Con esto en mente, defina $G_{i}^{\prime}=\left(\mathcal{U}_{i}, F_{i}^{\prime}\right)$ como el digrafo cuyos vértices son las componentes de $\mathcal{U}_{i}$ y donde para cada $U \in \mathcal{U}_{i}$ agregamos el arco $\tilde{U}=U f\left(e_{U}\right)$ a $F_{i}^{\prime}$. Por simplicidad extienda la función de costos $\operatorname{como} c(\tilde{U})=c\left(e_{U}\right)$
		\item Demuestre que los únicos diciclos de $G_{i}^{\prime}$ tienen largo 2 (corresponden a arcos paralelos). Más aún, pruebe que si $\tilde{U}$ y $\tilde{W}$ son arcos paralelos en $G_{i}^{\prime}$, entonces $e_{U}=e_{W}$.
		\item Sea $G_{i}^{\prime \prime}$ el grafo subyaciente de $G_{i}^{\prime}$ manteniendo solo una arista de cada par de arcos paralelos. Justifique que $G_{i}^{\prime \prime}$ no tiene ciclos. Use esto para demostrar inductivamente que $G_{i}=\left(V, F_{i}\right)$ no posee ciclos.
		\item Concluya.
	\end{enumerate}
	
	% \item \textbf{[Propuesto] Second-Best MST}
	
	% Sea $G=(V, E)$ un grafo conexo con peso en los arcos $w: E \rightarrow \mathbb{Q}_{+}$, suponga además que todos los pesos son distintos.
	% Sea $\mathcal{T}$ el conjunto de todos los árboles generadores de $G$ y $T^{*}$ el único MST, diremos que $T$ es un segundo-mejor MST si $w(T)=\min _{A \in \mathcal{T} \setminus \{T^*\}} w(A)$.
	% \begin{enumerate}
	% 	\item Demuestre que en este caso efectivamente el MST es único.
	% 	\item Sea $T^{*}$ el MST de $G$. Demuestre que hay aristas $e, f \in E$ tales que $T^{*}-e+f$ es un segundo-mejor MST.
	% 	\item Sea $T$ un ST de $G, u, v \in V$ y $P_{u, v}$ el único $u v$ camino entre $u$ y $v$ en $T$. Definimos:
	% 	\[\max [u, v]=\underset{e \in P_{u, v}}{\arg \max} w_e\]
	% 	Diseñe un algoritmo que dado $T$ calcule $\max [u, v]$ para todo par $u, v \in V$ en tiempo $O(|V|^2)$.
	% 	\item De un algoritmo $O(|V||E|)$ que dado un grafo $G$ calcule el segundo-mejor MST.
	% 	\begin{obs}
	% 		Es posible que encunetre un algoritmo $O\left(|V|^{2}\right)$.
	% 	\end{obs}
	% \end{enumerate}
\end{enumerate}
\end{document}