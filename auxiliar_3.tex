\documentclass[12pt]{article}

\usepackage[left=2cm,top=2cm,right=2cm, bottom=2cm]{geometry}
\usepackage[T1]{fontenc}
\usepackage[utf8]{inputenc}
\usepackage[spanish]{babel}
\usepackage{amsfonts,setspace}
\usepackage{amsmath}
\usepackage{amssymb, amsmath, amsthm}
\usepackage{comment}
\usepackage{amssymb}
\usepackage{dsfont}
\usepackage{anysize}
\usepackage{multicol}
\usepackage{enumerate}
\usepackage{graphicx}
\usepackage{fancyhdr}
\usepackage{wasysym}
\usepackage{enumerate}
\usepackage{enumitem}
\usepackage{hyperref}

\pagestyle{fancy}

\theoremstyle{plain}
\newtheorem{teo}{Teorema}
\newtheorem{lemma}{Lemma}
\newtheorem{prop}{Proposici\'on}
\newtheorem{cor}{Corolario}
\theoremstyle{definition}
\newtheorem{defi}{Definici\'on}
\newtheorem{eje}{Ejemplo}
\newtheorem{obs}{Observaci\'on}
\newtheorem{propi}{Propiedades}

\newcommand{\norm}[1]{\lVert#1\rVert}
\newcommand{\ds}{\displaystyle}
\newcommand{\R}{\mathbb{R}}

% Error fixes
\makeatletter
\newcommand\RedeclareMathOperator{%
  \@ifstar{\def\rmo@s{m}\rmo@redeclare}{\def\rmo@s{o}\rmo@redeclare}%
}
% this is taken from \renew@command
\newcommand\rmo@redeclare[2]{%
  \begingroup \escapechar\m@ne\xdef\@gtempa{{\string#1}}\endgroup
  \expandafter\@ifundefined\@gtempa
     {\@latex@error{\noexpand#1undefined}\@ehc}%
     \relax
  \expandafter\rmo@declmathop\rmo@s{#1}{#2}}
% this is just \@declmathop without \@ifdefinable
\newcommand\rmo@declmathop[3]{%
  \DeclareRobustCommand{#2}{\qopname\newmcodes@#1{#3}}%
}
\@onlypreamble\RedeclareMathOperator
\makeatother

\DeclareMathOperator{\sen}{sen}
\RedeclareMathOperator{\cos}{cos}
\RedeclareMathOperator{\tan}{tan}
\RedeclareMathOperator{\sec}{sec}
\DeclareMathOperator{\cosec}{cosec}
\DeclareMathOperator{\cotan}{cotan}
\DeclareMathOperator{\arcsen}{arcsen}
\RedeclareMathOperator{\arccos}{arccos}
\RedeclareMathOperator{\arctan}{arctan}

\DeclareMathOperator{\senh}{senh}
\RedeclareMathOperator{\cosh}{cosh}
\RedeclareMathOperator{\tanh}{tanh}
\DeclareMathOperator{\sech}{sech}
\DeclareMathOperator{\cosech}{cosech}
\DeclareMathOperator{\cotanh}{cotanh}
\DeclareMathOperator{\arcsenh}{arcsenh}
\DeclareMathOperator{\arccosh}{arccosh}
\DeclareMathOperator{\arctanh}{arctanh}

\DeclareMathOperator{\Dom}{Dom}
\DeclareMathOperator{\Rec}{Rec}
\RedeclareMathOperator{\Im}{Im}

\DeclareMathOperator{\Int}{Int}
\DeclareMathOperator{\Adh}{Adh}
\DeclareMathOperator{\Fr}{Fr}
\DeclareMathOperator{\co}{co}

\DeclareMathOperator{\dist}{dist}
\DeclareMathOperator*{\argmin}{arg\,min}
\DeclareMathOperator*{\argmax}{arg\,max}

\let\lim=\undefined\DeclareMathOperator*{\lim}{\text{lím}}
\let\max=\undefined\DeclareMathOperator*{\max}{\text{máx}}
\let\min=\undefined\DeclareMathOperator*{\min}{\text{mín}}
\let\inf=\undefined\DeclareMathOperator*{\inf}{\text{ínf}}

\newcommand{\pint}[2]{\left< #1,#2\right>}
\newcommand{\ssi}{\Longleftrightarrow}
\newcommand{\imp}{\Longrightarrow}
\newcommand{\pmi}{\Longleftarrow}
\newcommand{\ipartial}[2]{\dfrac{\partial #1}{\partial #2}}
\newcommand{\ider}[2]{\dfrac{d #1}{d #2}}
\newcommand{\iipartial}[2]{\dfrac{\partial^2 #1}{\partial #2^2}}
\newcommand{\iider}[2]{\dfrac{d^2 #1}{d #2^2}}
\newcommand{\ijpartial}[3]{\dfrac{\partial^2 #1}{\partial #2 \partial #3}}

\newcommand{\N}{\mathbb{N}}
\newcommand{\Z}{\mathbb{Z}}
\newcommand{\Q}{\mathbb{Q}}
\newcommand{\C}{\mathbb{C}}
\newcommand{\K}{\mathbb{K}}
\renewcommand{\P}{\mathbb{P}}

\newcommand{\A}{\mathcal{A}}
\newcommand{\B}{\mathcal{B}}
\newcommand{\D}{\mathcal{D}}
\newcommand{\E}{\mathcal{E}}
\newcommand{\F}{\mathcal{F}}
\renewcommand{\H}{\mathcal{H}}
\newcommand{\I}{\mathcal{I}}
\renewcommand{\L}{\mathcal{L}}
\newcommand{\M}{\mathcal{M}}
\newcommand{\Partes}[1]{\mathcal{P}(#1)}
\renewcommand{\S}{\mathcal{S}}
\newcommand{\Tau}{\mathcal{T}}
\newcommand{\V}{\mathcal{V}}
\newcommand{\X}{\mathcal{X}}
\newcommand{\vacio}{\emptyset}
\newcommand{\tsc}{\textsc}

\newcommand{\header}{
    \fancyhead[L]{Facultad de Ciencias Físicas y Matemáticas}
	\fancyhead[R]{Universidad de Chile}
	\vspace*{-1cm}
	\hspace{-0.5cm}\begin{minipage}{0.7\textwidth}
	\begin{flushleft}
	\textbf{MA3705 Algoritmos Combinatoriales}.\\
	\textbf{Profesor:} Iván Rapaport.\\
	\textbf{Auxiliares:} Antonia Labarca y Cristian Palma.\\
	\end{flushleft}
	\end{minipage}
	\begin{minipage}{0.3\textwidth}
	\begin{flushright}
	\hspace{-0.5cm}\includegraphics[scale=0.2]{img/dim.pdf}
	\end{flushright}
	\end{minipage}

	\medskip
}
\begin{document}
\header
\begin{center}
	\LARGE \bf{Auxiliar 3}
\end{center}

\begin{center}
	\bf{Árboles Generadores de Costo Mínimo y Matroides}\\
\end{center}

\begin{enumerate}[label ={\bf P\arabic*}]

	\item Considere $G=(V,E)$ conexo y $w:E\to \Q^+$ una función de peso. \begin{enumerate}
		%\item Sea $C\subseteq G$ un ciclo. Sea $e\in E(C)$ tal que $w(e)\geq w(f) \forall f \in E(C)$. Pruebe que existe $T^*$ MST de $G$ tal que $e\notin E(T^*)$.
		\item Sea $\emptyset \neq U\subset V$. Sea $e\in \delta(U)$ tal que $w(e)\leq w(f), \forall f \in \delta(U)$. Pruebe que existe $T^*$ MST de $G$ tal que $e\in E(T^*)$.
		\item Sea $T^*$ un MST de $G$. Sea $e=uv\in E(T)$. Sea $P$ un $u$-$v$-camino en $G-e$. Pruebe que $w(e) \leq w(f), \forall f \in E(P)$
	\end{enumerate}
	
	\item Para $G=(V,E)$ grafo conexo se define \textbf{Minimum Bottleneck Spanning Tree}\footnote{También lo pueden encontrar como MinMax Spanning Tree.} como un árbol generador de $G$ en el que la arista de mayor costo es lo más barata posible.\begin{enumerate}
		\item Pruebe que si $T$ es MST de $G$ también es MBST.
		\item ¿Se puede asegurar que si $T$ es MBST entonces también es MST?
	\end{enumerate}
	
	\item Considere $\mathcal{M}=(\mathcal{S}, \mathcal{I})$ una matroide. Se define un \textbf{circuito} como un conjunto dependiente minimal, es decir, un conjunto $C$ tal que $C\notin \mathcal{I}, \forall e \in C, C-e\in\mathcal{I}$.\begin{enumerate}
		\item Sean $C,D$ circuitos no disjuntos y sea $e\in C\cap D$. Pruebe que $C\cup D - e$ no es independiente.
		\item Sea $I$ independiente y sea $e\notin I$. Pruebe que $I+e$ tiene a lo más 1 circuito.
	\end{enumerate}

	\item Sea $\tilde{\mathcal{M}}=(\tilde{\mathcal{S}},\tilde{\mathcal{I}})$. Diremos que $\tilde{\mathcal{M}}$ es \textbf{de partición} si existe una partición de $\tilde{\mathcal{S}}$ en conjuntos $S_1, \dots , S_k$ no vacíos (llamados bloques) y existen $b_1, \dots , b_k\in\N$ tales que $\forall X\subseteq \tilde{\mathcal{S}}\hspace{0.5em} (X\in \tilde{\mathcal{I}}\iff \forall i\in[k], X\cap S_i \leq b_i)$.
		\\ Pruebe que si $\tilde{\mathcal{M}}$ es de partición, entonces es una matroide.

	\item Dado $G=(V,E)$ grafo conexo, se define $I(G):=\{E'\subseteq E | (V, E \setminus E') \text{ es conexo}\}$.
		\\ Pruebe que $M(G) = (E, I(G))$ es matroide\footnote{Se llama matroide cográfica de $G$}.

\end{enumerate}
\end{document}